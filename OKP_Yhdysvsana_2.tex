\documentclass{tktltiki}

\usepackage[pdftex]{graphicx}
\usepackage{subfigure}
\usepackage{url}
\usepackage{amsmath}
\usepackage{comment}

\begin{document}
\doublespacing
%\singlespacing
%\onehalfspacing

\title{Otsikko}
\author{Erkki Heino \and Tero Huomo \and Eeva Terkki}
\date{\today}

\maketitle


\mytableofcontents

\section{N�kyvyysalueet}

\subsection{COBOL}

\subsection{Python}

Pythonin lohkorakenne on syv�, ja ohjelman suoritusaikana k�yt�ss� on ainakin kolme sis�kk�ist� n�kyvyysaluetta \cite{pythonclasses}. N�kyvyysalueita k�ytet��n dynaamisesti. Sisimm�ll� n�kyvyysalueella ovat paikalliset nimet. Mahdollisilla funktioita ymp�r�ivill� funktioilla on omat n�kyvyysalueensa, joiden sis�lt�m�t nimet eiv�t ole paikallisia eiv�tk� globaaleja. Toisiksi uloimmalla n�kyvyysalueella ovat moduulin globaalit nimet ja kaikkein uloimmalla kieleen rakennetut nimet.

\begin{verbatim}
(esimerkki)
\end{verbatim}

Pythonissa kaikki asiat, jotka voidaan nimet�, ovat ensimm�isen luokan arvoja -- my�s funktiot, metodit ja moduulit \cite{pythonhistory}. 

\newpage
 
\bibliographystyle{tktl}
\bibliography{lahteet}

\lastpage

\appendices

\pagestyle{empty}

%\internalappendix{1}{Malli ABC}


\end{document}
