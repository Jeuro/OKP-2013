\documentclass{tktltiki}

\usepackage[pdftex]{graphicx}
\usepackage{subfigure}
\usepackage{url}
\usepackage{amsmath}
\usepackage{comment}
\usepackage{caption}

\begin{document}
%\doublespacing
%\singlespacing
\onehalfspacing

\title{COBOL ja Python: Datan kapselointi}
\author{Erkki Heino, Tero Huomo, Eeva Terkki}
\date{\today}

\maketitle


\mytableofcontents

\section{Datan kapselointi}

\subsection{COBOL}

COBOLissa tietoa voidaan ryhmitell� muodostamalla tietueita (record) ja ryhmi� (group). \verb+DATA DIVISION+ -osiossa k�ytett�v�t muuttujien tasonumerot (level number) m��ritt�v�t, mink�laisia kokonaisuuksia muuttujat muodostavat.



\begin{samepage}
\begin{verbatim}
    DATA DIVISION.
    WORKING-STORAGE SECTION.
    01 OPISKELIJA.
       05 NIMI.
           10 ETUNIMI  PIC A(20).
           10 SUKUNIMI PIC A(20).
       05 NUMERO       PIC 9(15).
       05 SUKUPUOLI    PIC A.
           88 MIES     VALUE "M".
           88 NAINEN   VALUE "N".
\end{verbatim}
\end{samepage}

\subsection{Python}



\section{Etuja ja haittoja}


\newpage
 
\bibliographystyle{tktl}
\bibliography{lahteet}

\lastpage

\appendices

\pagestyle{empty}

%\internalappendix{1}{Malli ABC}

\end{document}