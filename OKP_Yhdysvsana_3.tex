\documentclass{tktltiki}

\usepackage[pdftex]{graphicx}
\usepackage{subfigure}
\usepackage{url}
\usepackage{amsmath}
\usepackage{comment}

\begin{document}
%\doublespacing
%\singlespacing
\onehalfspacing

\title{COBOL ja Python: Tyypitys ja laskennan kapselointi}
\author{Erkki Heino, Tero Huomo, Eeva Terkki}
\date{\today}

\maketitle


\mytableofcontents

\section{Tyyppij�rjestelm�t}

\subsection{COBOL}

\subsection{Python}

Python on vahvasti tyypitetty kieli \cite{staticvsdynamic}. Tietyn tyyppiselle muuttujalle ei voida tehd� toisen tyypin operaatioita ennen eksplisiittist� tyyppimuunnosta.

\begin{samepage}
\begin{verbatim}
a = 5
b = "9"
c = a + int(b)
\end{verbatim}
\end{samepage}

Esimerkiss� \verb+b+ sis�lt�� merkkijonon "9", mutta yhteenlaskussa merkkijonosta j�sennet��n kokonaisluku. Jos kokonaislukuj�sennyksen j�tt�� tekem�tt�, antaa ohjelma kyseisell� rivill� poikkeuksen. 

Suoritusaikana muuttujan tyyppi ei ole sidottu, vaan muuttujaan voi dynaamisesti sitoa eri vaiheessa eri tyyppisi� olioita. Seuraavassa esimerkiss� muuttuja \verb+a+ saa ensin kokonaislukuarvon 5. Sen j�lkeen muuttujan \verb+a+ arvoksi muutetaan merkkijono "hei".

\begin{samepage}
\begin{verbatim}
a = 5
a = "Hei"
\end{verbatim}
\end{samepage}

\section{Laskennan kapselointi}

\subsection{COBOL}

\subsection{Python}

\section{Etuja ja haittoja}

\newpage
 
\bibliographystyle{tktl}
\bibliography{lahteet}

\lastpage

\appendices

\pagestyle{empty}

%\internalappendix{1}{Malli ABC}

\end{document}